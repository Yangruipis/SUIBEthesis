\documentclass{SUIBEthesis}

\usepackage{booktabs}
\usepackage{cite}
\usepackage[round, comma, sort&compress]{natbib}
\usepackage[nottoc]{tocbibind}        % 将参考文献加入目录
\bibliographystyle{plain}    % 引用文献格式


\renewcommand{\baselinestretch}{1.5}


\begin{document}

\title{上市公司投资网络:风险传染还是风险弱化}
\author{杨睿}
\studentnum{16260715}
\advisor{周晓东~~副教授}
\degree{科学学位}
\school{统计与信息学院}
\major{数量经济学}
\finishdate{2018年11月}


\maketitle

\newpage

\begin{abstract}
  腐败问题一直是影响国计民生的重要话题,如何公平、有效的治理腐败也是学界关切的核心议题之一。本研究将2015年法律文书网站OpenLaw上的裁判文书数字化,探讨了腐败行为、官员职级之间的影响关系和惩罚公平的议题。

  回归分析的结果表明,在控制其他变量不变的前提下,官员的裁定受贿金额和最终量刑刑期呈对数关系,且二者间边际递减模式显著。在这种模式下,虽然官员的职级与似乎并不影响最终刑期关系,但是如果考虑法官的自由裁量权,官员的职级便与自由裁量下的溢出刑期呈显著正U型模式,即级别越高的官员和级别越低的官员相对的被量刑较重,级别靠近副科级左右的被告人是相对量刑最轻的。而且这种不公平的惩罚模式是与地区经济水平无关的。如果考虑到不同单位之间的异质性,国家机关、事业单位和国有企业中均存在被告人级别与溢出刑期的正U型曲线关系,且在级别相同的情况下,事业单位和国有企业中的惩罚较国家机关中的更轻。在考虑到样本缺失、删失数据和极端值的影响因素后,上述基本模式依然稳健。最后,本研究给出了规范自由裁量权、提高量刑标准等合理建议。
\end{abstract}

\keywords{腐败;计量}

\newpage
\begin{enabstract}
  Corruption has always been an important topic affecting the people's livelihood, how to tackle corruption fairly and effectively is also one of the core topic in academia. This study makes a detailed quantitative analysis of the relationship among corruption, the rank of officials and the fairness of punishment by digitizing the court verdicts in 2015 on OpenLaw, an online legal document website.

  
The results of the regression show that in the case of control other variables the more amount of logarithmic bribery has increased the more final sentence, there is a clear marginal decreasing pattern between them. There is no significant relationship between the rank of the officials and final sentences under this pattern. However, if the judge's discretionary power is taken into account, the quadratic rank of the officials and the spillover term of penalty are obviously significant. That is, the higher rank and the lower rank of the officials are punished relatively heavier, the Vice-county officials have a lightest punishment, and his unfair punishment has no relationship with the local economics.If the heterogeneity among different units is taken into account, it is found that there is a positive U-shaped pattern between the defendant's rank and the spillover term of penalty in government sectors, public institutions and state-owned enterprises. However, when difference of rank is controlled, the punishment in public institutions and state-owned enterprises is lighter than the government sectors. The patterns mentioned above are very robust, even in consideration of the problem of missing samples, censored data and extreme values.


At last, some suggestions like standardizing the discretionary power and raising the standard of sentencing are given.

\end{enabstract}

\enkeywords{Corruption, Rank, Discretionary power, U-shaped curve}


% ------------------- 目录页 -----------------------
\newpage
\pagestyle{empty}
\tableofcontents
\pagestyle{empty}
\thispagestyle{coverpage}
% --------------------------------------------------


\newpage
\setcounter{page}{1}
\pagestyle{mainpage}





\section{测试一级章节}
\subsection{测试二级章节}
\subsubsection{测试三级章节}

测试脚注\footnote{I love \LaTeX}

测试参考文献引用:

\begin{itemize}
\item 一般引用\cite{lamport1994latex}
\item 多重引用\cite{lamport1994latex, knuth1984texbook}
\item 其他类型\cite[see][]{knuth1984texbook}
\item 其他类型\cite[section2]{knuth1984texbook}
  \item \cite[see][]{knuth1984texbook, lamport1994latex}
\end{itemize}


正文字体字号行距测试正文字体字号行距测试正文字体字号行距测试正文字体字号行距测试正文字体字号行距测试正文字体字号行距测试正文字体字号行距测试正文字体字号行距测试正文字体字号行距测试正文字体字号行距测试正文字体字号行距测试正文字体字号行距测试正文字体字号行距测试正文字体字号行距测试正文字体字号行距测试正文字体字号行距测试

正文字体字号行距测试正文字体字号行距测试正文字体字号行距测试正文字体字号行距测试正文字体字号行距测试正文字体字号行距测试正文字体字号行距测试正文字体字号行距测试正文字体字号行距测试正文字体字号行距测试正文字体字号行距测试正文字体字号行距测试正文字体字号行距测试正文字体字号行距测试正文字体字号行距测试


测试图片~\ref{fig:1} 

\begin{figure}[htbp!]
  \centering
  \includegraphics[height=3cm]{data/figure/logo.pdf}
  \caption{上海对外经贸大学校徽}
  \label{fig:1}
\end{figure}


测试表格~\ref{tab:1}

\begin{table}[htbp]
  \centering
  \caption{网络节点与边数统计}
  \begin{tabular}{ccc}
    \toprule
    \toprule
    年份    & 节点(上市公司)数   & 边数 \\
    \midrule
    2005  & 978   & 5,736 \\
    2006  & 1,002 & 5,691 \\
    2007  & 1,081 & 6,014 \\
    2008  & 1,141 & 6,132 \\
    \bottomrule
    \bottomrule
  \end{tabular}%
  \label{tab:1}%
\end{table}%

公式测试:

\begin{eqnarray}
L(y_i, f(x_i)) = \frac{1}{n}\sum_{i=1}^n y_i \ln (\sigma(x_i)) + (1-y_i)\ln (1-\sigma(x_i))
\end{eqnarray}



\section{测试一级章节2}
测试脚注2\footnote{这是第二个脚注}

\begin{eqnarray}
a^2 + b^2 = c^2
\end{eqnarray}

\newpage
\xiaosi

\bibliography{ref}

\newpage
\begin{mythanks}

感谢我的老师。。。
\end{mythanks}

\end{document}


